\documentclass[a4paper,12pt,oneside,notitlepage,onecolumn]{article}

\usepackage{ucs}
\usepackage[utf8x]{inputenc}

\usepackage{fontenc}
\usepackage{graphicx}

\usepackage[OT4]{fontenc}
\usepackage[polish]{babel}
\usepackage{polski}
\usepackage{indentfirst}
\usepackage{graphics}

\usepackage[dvips]{hyperref}

\author{Michał Bobowski, Zofia Abramowska, Jakub Meller}
\title{Dokumentacja do projektu AgentChess}

\begin{document}
  \maketitle
\section{Założenia wstępne}
Niniejszy projekt jest realizowany w ramach przedmiotu SAG.
Jego przedmiotem jest stworzenie prostego, bazującego na agentowym podejściu, modelu sztucznej inteligencji służącej do gry w szachy.
Silnik gry i warstwa graficzna zostaną oparte na istniejących rozwiązaniach.
Całość zostanie zrealizowana w języku programowania Scala.

\subsection{Figury}
Dla uporządkowania wiedzy dziedzinowej, posługujemy się nazwami figur w wesji polskiej i angielskiej.
Są to kolejno:
\begin{itemize}
 \item Pion (ang. Pawn) - porusza się o jedno pole naprzód (ew. o dwa pola na początku), bije tylko po skosie.
 \item Skoczek (ang. Knight) - porusza się systemem dwa pola naprzód i jedno do boku w dowolnym kierunku.
 \item Goniec (ang. Bishop) - porusza się po skosie.
 \item Wieża (ang. Rook) - porusza się w pionie lub poziomie.
 \item Hetman (ang. Queen) - łączy ruchy wieży i gońca.
 \item Król (ang. King) - porusza się o jedno pole w dowolnym kierunku.
\end{itemize}

Pion po dojściu do końca planszy może zamienić się w inną figurę.
Nie obsługujemy od strony AI roszady i bicia w przelocie.

\section{Uwagi techniczne - przeczytaj zanim zaczniesz kodować}
Chciałbym zwrócić w tej sekcji uwagę na kilka problemów z którymi miałem do czynienia.
Po pierwsze - istnieje duża różnica między bibliotekami Scala Actors i Scala Akka, choć na pierwszy rzut oka wyglądają tak samo.
My korzystamy TYLKO z biblioteki Scala Akka, która od wersji języka 2.11 zastąpi Scala Actors.
Ponieważ najnowszym wydaniem stabilnym języka Scala jest wersja 2.10 to musiałem ręcznie dodać sporo plików jar do java build path (cały folder /lib w repozytorium).

Korzystałem z IDE Eclipse z wtyczką do języka Scala.
Nie mogę powiedzieć, żeby ten zestaw działał oszałamiająco dobrze, ale na potrzeby tego projektu wystarcza.
Trzeba się tylko przyzwyczaić, że czasem pokazuje błędy tam gdzie ich nie ma i dopiero \emph{Project/Clean} załatwia sprawę.

Długo zastanawiałem się jak najlepiej wykorzystać istniejące szachy z książki Grzegorza Balcerka.
Architektura, którą przyjął autor nie była dla nas z wielu względów korzystna np. logika ruchu poszczególnych figur nie znajdowała się bezpośrednio w klasach tych figur.
Z drugiej strony głupio było od nowa pisać interfejs graficzny.
Ostatecznie zdecydowałem się na użycie podstawowych klas (Figure, Field, Game), ale logikę ruchu napisałem jeszcze raz w klasach agentów.
W ten sposób ruchy gracza są wykonywane tak jak w oryginale.

Ciekawe jest to, że gra w szachy wcale nie kończy się zbiciem króla.
Ruchy, które pozostawiają własnego króla w sytuacji zagrożenia, nie są w ogóle akceptowane jako poprawne.
Dlatego nasze AI ogłasza przegraną, kiedy nie może wykonać żadnego poprawnego ruchu.

\section{Architektura agentów}
Agenci są umieszczeni w architekturze wertykalnej.
Najważniejsze decyzje podejmowane są przez super-agenta, którego zdaniem jest wysyłanie żądań do agentów podrzędnych oraz analiza ich odpowiedzi.

Na niższym poziomie w hierarchii znajdują się agenci reprezentujący figury obsługiwane przez nasze AI.
Wykonują oni niezależnie od siebie zadania obliczeniowe i wysyłają ich wyniki do nadzorcy.

W trakcie kodowania okazało się, że potrzebna jest jeszcze jedna warstwa, łącząca super-agenta z dotychczasowym szkieletem gry.
Jest nią klasa \emph{Listener}, będąca klasą wewnętrzną \emph{ChessGui}.

\section{Struktury pomocnicze algorytmu}
W tej sekcji zawarty jest opis pewnych abstrakcyjnych bytów, biorących udział we wnioskowaniu.

\subsection{Ocena heurystyczna}
Każda z figur biorących udział w grze posiada heurystyczną ocenę swojej przydatności, oznaczaną dalej jako H.
Przyjęliśmy rozwiązanie proponowane przez wikipedię:
\begin{itemize}
 \item H=1 dla piona
 \item H=3 dla skoczka i gońca
 \item H=5 dla wieży
 \item H=9 dla hetmana
\end{itemize}

Do kompletu dodany zostaje król z wyceną równą np. H=20.

\subsection{Mapa bicia wroga}
Mapa bicia wroga pokazuje, które figury AI mogą zostać zbite w następnym wykonywanym przez gracza ruchu.
Kluczem mapy jest pole planszy, a wartością zmienna boolowska, która jest równa \emph{true} wtedy i tylko wtedy gdy figura należąca do gracza może wykonać ruch na dane pole.
Oznacza to dla figur AI możliwość zbicia na każdym polu oznaczonym jako \emph{true}.

W mapie tej dodano dodatkową informację - czy dane pole zajmowane przez figurę przeciwnika jest wspierane przez inną figurę(y).
Jest to istotne z punktu widzenia figury bijącej, gdyż może oznaczać natychmiastową "zemstę" ze strony przeciwnika.
Uzyskano to poprzez rozszerzenie warunku oznaczania pól w mapie bicia na \emph{true} - w momencie napotkania przyjaznej figury oznaczanie zostaje zakończone,
lecz w przeciwieństwie do zwykłej mapy bicia, pole jest też ustawione na \emph{true}.
W ten sposób łączy się informację o polu bitym oraz o polu wspieranym.  

\section{Aktualizacja stanu}
Na początku gry tworzeni są agenci odpowiadający za każdą z figur AI.
Życie agentów kończy się w bliżej nieokreślonej przyszłości, po tym jak zostaną oni zbici przez gracza.
Z tego powodu, działanie algorytmu rozpoczyna się od sprawdzenia, czy poprzedni ruch gracza nie zakończył się na polu zajmowanym przez jedną z figur AI.
Odpytanie jest wykonywane przy pomocy komunikatów.
W przypadku stwierdzenia swojego zgonu, agent figury przesyła stosowny komunikat do super-agenta i kończy swoje działanie.

\section{Faza pierwsza - odpytanie o ruchy}
Następną fazę algorytmu rozpoczyna super-agent, który zadaje każdemu agentowi podrzędnemu pytanie o możliwe ruchy.
Odpowiedzią na żądanie jest lista ruchów, które mogą być wykonane zgodnie z zasadami gry.
Bicie jest traktowane jako normalny ruch.

Struktura danych reprezentująca ruch zawiera następujące dane:
\begin{itemize}
 \item Pole początkowe (source)
 \item Pole końcowe (destination)
 \item Wstępną ocenę heurystyczną ruchu
\end{itemize}

Wstępna ocena heurystyczna ma domyślnie wartość 0.
Jedynie w przypadku bicia przyjmuje wartość równą wartości ofiary.

\section{Faza druga - ocena ruchów}
Każdy ruch podlega ocenie przez wszystkich agentów, wliczając w to agenta zmieniającego położenie.
Faza oceny pojedynczego ruchu przebiega według następującego schematu:
\begin{enumerate}
 \item Super-agent odtwarza stan planszy po wykonaniu ruchu
 \item Super-agent buduje dla nowego stanu mapę ruchu wroga
 \item Super-agent wysyła do agentów podrzędnych pytanie o subiektywną ocenę swojego położenia
\end{enumerate}

Agenci podrzędni wyrażają swoją ocenę sytuacji w sposób egoistyczny tzn. nie interesują się pozostałymi agentami.
Ocena sytuacji jest wyrażana liczbowo i jest wypadkową dwóch składowych: ofensywnej i defensywnej.

Jeżeli nasza figura jest w danej sytuacji zagrożona, to ocena jest równa ujemnej wartości H naszej figury.
W przeciwnym przypadku ocena jest równa najwyższej ocenie figury wroga, znajdującej się w zasięgu rażenia naszej figury.
Jeżeli figura wroga jest wspierana, to składową ofensywną pomniejszamy o wartość naszej figury.

Końcowa ocena ruchu jest sumą następujących składowych:
\begin{enumerate}
 \item Oceny z fazy pierwszej
 \item Najniższej z ocen agentów
 \item Średniej ze wszystkich ocen agentów
\end{enumerate}

\section{Faza trzecia - wybór ruchu}
Po skompletowaniu ocen cząstkowych, super-agent przesyła do obiektu nasłuchującego, posortowaną względem sumarycznej oceny listę ruchów.
Następnie wykonywany jest najlepszy z ruchów, który nie pozostawia króla w pozycji zagrożenia.
Jeżeli żaden z ruchów nie spełni tego warunku, to AI ogłosi swoją porażkę.

Po wyborze ruchu następuje jego ogłoszenie do wiadomości wszystkich agentów.
Agent, który jest źródłem ruchu dokonuje aktualizacji swojego położenia.

\section{Wnioski}
Zaimplementowany model sztucznej inteligencji działa w sposób zgodny z oczekiwaniami.
Jeżeli gracz popełni błąd, wystawiając jedną ze swoich figur, to AI na pewno to wykorzysta.
Bardzo rzadko zdarzają się ruchy, których nie da się w logiczny sposób wytłumaczyć.

Wydajność algorytmu również prezentuje się zadowalająco.
Na średniej jakości komputerze stacjonarnym (Dual Core 2600MHz, 4GB RAM) wnioskowanie trwa mniej niż pięc sekund.

Niewątpliwą wadą sztucznej inteligencji jest brak zaprogramowanych schematów rozpoczęcia partii, czego konsekwencją jest wykonywanie losowych ruchów.
Doświadczony gracz na pewno uzyska w pierwszych kilku ruchach dużą przewagę.
Z drugiej strony przy mniejszej liczbie figur, AI staje się całkiem skuteczne w pogoni za wrogim królem.

\end{document}
